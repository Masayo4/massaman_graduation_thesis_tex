% ■ アブストラクトの出力 ■
%	◆書式:
%		begin{jabstract}〜end{jabstract}	:日本語のアブストラクト
%		begin{eabstract}〜end{eabstract}	:英語のアブストラクト
%		※ 不要ならばコマンドごと消せば出力されない。



% 日本語のアブストラクト
\begin{jabstract}
人が暮らす社会は人と人との繋がりで構成されている.人間関係を構築する際には
コミュニケーションが必要であり, その中でも表情などの非言語コミュニケーション,
特に笑顔は大きく影響を及ぼす.
人との繋がり形成の場が多様化し,入り口が広くなった分情報量が多くなり,整理がうまく行えていない現状がある.若者がテキストベースのやりとりを行った後, 実際に会った際には相手にがっかりするケースが非常に多くなっている.
本研究の目的は笑顔の作り方からお互いを分析し, 人と人との繋がりを助長するシステムの構築をである.
自身の笑い方と, 魅力を感じる笑い方にはどのような関係性があるのかを考察し, 嗜好傾向を分析する.
データに基づく分析を行うことで, より最適な相手をユーザーに表示し, よりよい出会いの機会提供を助長することを目指す.
%<位置変更>
本研究ではユーザーの中立の表情から,笑顔になる過程を画像処理し,数値化するシステムを作成する.
表情の特徴点を示した動画に対してユーザーに順位づけをさせ, ユーザー自身の笑顔の作り方と表情ベースにおけるユーザーごと,全ユーザーに共通した嗜好傾向の分析を行った.

慶應義塾大学湘南藤沢キャンパス主催のOpen Reserch Forumにてデモンストレーションおよび評価実験を行い, 41人の来場者のデータを取得した. 破損データを除く36個の笑顔動画データからCambridge大学が開発したOpenFaceに含まれる,Facial Action Units(FAU)の値を算出しユーザーごとに嗜好傾向分析を行った.
分析の結果, 人は一度顔の筋肉を弛緩してから笑顔になるユーザーに嗜好傾向があることが判明した.

%<変更>
今後の展望として, 本システムはより多くのデータを収集, ユーザーの内面状態を組み込むことで
仕事または結婚時のパートナー選択に有益な情報を提供することが可能である.
情報過多になっている現代社会において, ユーザーにとって適切なデータ提供を行い,
より良い人間関係の構築をする機会を提供することが可能になると考えられる.
本研究が, 人と人との良縁を結ぶ役割を担うようなシステムになることを期待する.

\end{jabstract}

% 英語のアブストラクト
\begin{eabstract}


  The society we live in is constructed from the connections between the people.
  In order to establish this relationship, communication is crucial, especially non-verbal communication,
  and factors such as smiling can have significant influences.
  The way in which people connect have become diverse, and with this, more information is available;
  however this information has still yet to be organized.
  In the younger generations,
  many may often be disappointed when actually meeting the people that they had previously “met” online via text.
  The purpose of this research is to build a system that promotes the connection between people based on analyzing each other based on their smiles.
  This system finds a correlation between one’s smile and the smile that they prefer, and analyzes the inclination.
  By performing data-based analysis, I aim to make connect the most optimal partners to users, and encourage better encounters.

  In this research we create a system that digitizes the process in which the users smile from a neutral expression, via image processing.
  Based on the calculated values,
  the system displays videos from the database of the process in which people smile.
  These videos show the featured points in place of the person’s face.
  The users then rank the videos, after which the analysis is conducted on each user,
  and the users as a group, on how the users smile, and on their smile preferences.

  I conducted demonstrations and evaluation experiments at the Open Research Forum hosted by Keio University Shonan Fujisawa Campus,
  and obtained datasets from 41 visitors.
  I got the value of Facial Action Units (FAU) included in OpenFace developed by Cambridge University from 36 smile video data excluding damaged data,
  and analyzed the inclination tendency for each users.
  As a result of the analysis, it was found that people tend to prefer people who relax their facial muscles before smiling.

In the future,
it will be possible to provide beneficial information on partner selection at work or marriage,
when the system will collect more data and consider users' insides.
In today's information overload society, this system provides users with appropriate data,
It will be possible to provide an opportunity to build better human relationships.
I expect that this system will play a critical role in connecting people.

\end{eabstract}

\begin{comment}
<修正前>
本研究ではユーザーの中立の表情から,笑顔になる過程を画像処理し,数値化するシステムを作成する.
算出した値を元にデータベースにある笑顔動画データをユーザーに人物が特定できないように, 表情の特徴68点の動きのみを5人分表示する.
選ばれた特徴点を示した動画に対してユーザーに順位づけをさせ, ユーザー自身の笑顔の作り方と表情ベースにおけるユーザーごと,
全ユーザーに共通した嗜好傾向の分析を行った.
人が暮らす社会は人と人との繋がりで構成されている.人間関係を構築する際には
コミュニケーションが必要であり, その中でも表情などの非言語コミュニケーション,
特に笑顔は大きく影響を及ぼす.
人との繋がり形成の場が多様化し,入り口が広くなった分情報量が多くなり,整理がうまく行えていない現状がある.若者がテキストベースのやりとりを行った後, 実際に会った際には相手にがっかりするケースが非常に多くなっている.
本研究の目的は笑顔の作り方からお互いを分析し, 人と人との繋がりを助長するシステムの構築をである.
自身の笑い方と, 魅力を感じる笑い方にはどのような関係性があるのかを考察し, 嗜好傾向を分析する.
データに基づく分析を行うことで, より最適な相手をユーザーに表示し, よりよい出会いの機会提供を助長することを目指す.
慶應義塾大学湘南藤沢キャンパス主催のOpen Reserch Forumにてデモンストレーションおよび評価実験を行い, 41人の来場者のデータを取得した. 破損データを除く36個の笑顔動画データからCambridge大学が開発したOpenFaceに含まれる,Facial Action Units(FAU)の値を算出しユーザーごとに嗜好傾向分析を行った.
分析の結果, 人は一度顔の筋肉を弛緩してから笑顔になるユーザーに嗜好傾向があることが判明した.
今後の展望として, 中澤研究室が関与する健康情報コンソーシアムのTeamSmileで作成したSmileMeterに本システムのモジュールを組み込むことでより幅広く多くのデータの取得を可能にし,
人と人を繋ぐ役割を担うようなシステムになることを期待する.
\end{comment}

\begin{comment}
In this research we create a system that digitizes the process in which the users smile from a neutral expression, via image processing.
Based on the calculated values,
the system displays five videos from the database of the process in which people smile.
These five videos show the 68 featured points in place of the person’s face
so that the users’ cannot identify the person from the video.
The users then rank the videos, after which the analysis is conducted on each user,
and the users as a group, on how the users smile, and on their smile preferences.
The society we live in is constructed from the connections between the people.
In order to establish this relationship, communication is crucial, especially non-verbal communication,
and factors such as smiling can have significant influences.
The way in which people connect have become diverse, and with this, more information is available;
however this information has still yet to be organized.
In the younger generations,
many may often be disappointed when actually meeting the people that they had previously “met” online via text.
The purpose of this research is to build a system that promotes the connection between people based on analyzing each other based on their smiles.
This system finds a correlation between one’s smile and the smile that they prefer, and analyzes the inclination.
By performing data-based analysis, I aim to make connect the most optimal partners to users, and encourage better encounters.
I conducted demonstrations and evaluation experiments at the Open Research Forum hosted by Keio University Shonan Fujisawa Campus,
and obtained datasets from 41 visitors.
I got the value of Facial Action Units (FAU) included in OpenFace developed by Cambridge University from 36 smile video data excluding damaged data,
and analyzed the inclination tendency for each users.
As a result of the analysis, it was found that people tend to prefer people who relax their facial muscles before smiling.
In the future, TeamSmile in health information consortium related with Nakazawa Laboratory,
will incorporate the module of this system into the SmileMeter to enable the acquisition of more and more data.
I expect that this system will play a critical role in connecting people.

\end{comment}
