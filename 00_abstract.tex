% ■ アブストラクトの出力 ■
%	◆書式:
%		begin{jabstract}〜end{jabstract}	:日本語のアブストラクト
%		begin{eabstract}〜end{eabstract}	:英語のアブストラクト
%		※ 不要ならばコマンドごと消せば出力されない。



% 日本語のアブストラクト
\begin{jabstract}
本研究はユーザーの笑顔の作り方から画像処理を行い,表情の作り方を数値化するシステムを作成する.
算出した値を元にデータベースにある笑顔動画データをユーザーに人物が特定できないように, 表情の動きのみを5つ表示する.
選ばれた動画に対してユーザーに順位づけをさせ, ユーザー自身の笑顔の作り方と表情ベースにおけるユーザーごと,
全ユーザーに共通した嗜好傾向の分析を行った.
人が暮らす社会は人と人との繋がりで構成されている.人間関係を構築する際には
コミュニケーションが必要であり, その中でも表情などの非言語コミュニケーション,
特に笑顔は大きく影響を及ぼす.
人との繋がり形成の場が多様化し,入り口が広くなった分情報量が多くなり,整理がうまく行えていない現状がある.若者がテキストベースのやりとりを行った後, 実際に会った際には相手にがっかりするケースが非常に多くなっている.
本研究の目的は笑顔からお互いを分析し, 人と人との繋がりを助長するシステムの構築をである.
自身の笑い方と, 魅力を感じる笑い方にはどのような関係性があるのかを考察し, 嗜好傾向を分析する.
データに基づく分析を行うことで, より最適な相手をユーザーに表示し, よりよい出会いの機会提供を助長することを目指す.
慶應義塾大学湘南藤沢キャンパス主催のOpen Reserch Forumにてデモンストレーションおよび評価実験を行い, 41人の来場者のデータを取得した. 破損データを除く36個の笑顔動画データからCambridge大学が開発したOpenFaceに含まれる,Facial Action Units(FAU)の値を算出しユーザーごとに嗜好傾向分析を行った.
分析の結果, 人は一度顔の筋肉を弛緩してから笑顔になるユーザーに嗜好傾向があることが判明した.
今後の展望として, 中澤研究室が関与する健康情報コンソーシアムのTeamSmileで作成したSmileMeterに本システムのモジュールを組み込むことでより幅広く多くのデータの取得を可能にし,
人と人を繋ぐ役割を担うようなシステムになることを期待する.
\end{jabstract}



% 英語のアブストラクト
\begin{eabstract}

abstract\\
English is very very very difficult for me.
I can only use "massamanglish".
That is "99\% body language, and 1 \% passion".

\end{eabstract}
