% 独自のコマンド

% ■ アブストラクト
%  \begin{jabstract} 〜 \end{jabstract}  :日本語のアブストラクト
%  \begin{eabstract} 〜 \end{eabstract}  :英語のアブストラクト

% ■ 謝辞
%  \begin{acknowledgment} 〜 \end{acknowledgment}

% ■ 文献リスト
%  \begin{bib}[100] 〜 \end{bib}


\newif\ifjapanese

\japanesetrue  % 論文全体を日本語で書く(英語で書くならコメントアウト)

\ifjapanese
  %\documentclass[a4j,twoside,openright,11pt]{jreport} % 両面印刷の場合。余白を綴じ側に作って右起こし。
  \documentclass[a4j,11pt]{jreport}                  % 片面印刷の場合。
  \renewcommand{\bibname}{参考文献}
  \newcommand{\acknowledgmentname}{謝辞}
\else
  \documentclass[a4paper,11pt]{report}
  \newcommand{\acknowledgmentname}{Acknowledgment}
\fi
\usepackage{thesis}
\usepackage{ascmac}
\usepackage{graphicx}
\usepackage{multirow}
\usepackage{url}

\usepackage{listings}
\usepackage{dirtree}
\usepackage{color}
\usepackage{comment}


\lstset{
    basicstyle={\ttfamily\small}, %書体の指定
    frame=tRBl, %フレームの指定
    framesep=10pt, %フレームと中身(コード)の間隔
    %breaklines=true, %行が長くなった場合の改行
    linewidth=12cm, %フレームの横幅
    lineskip=-0.5ex, %行間の調整
    tabsize=2 %Tabを何文字幅にするかの指定
}
\setlength\floatsep{0pt} %dblfloatsep


%\bibliographystyle{jplain}
\bibliographystyle{unsrt}

%\bindermode  % バインダー用余白設定

% 日本語情報(必要なら)
\jclass  {卒業論文}                             % 論文種別
\jtitle    {ユーザーの笑顔時における特徴点分析を用いた\\嗜好判断システムの構築}    % タイトル。改行する場合は\\を入れる
\juniv    {慶應義塾大学}                  % 大学名
\jfaculty  {環境情報学部 環境情報学科}               % 学部、学科
\jauthor  {鶴岡 雅能}                       % 著者
\jadvisorcaption {指導教員}
\jadvisor {中澤 仁\\村井 純\\楠本 博之\\中村 修\\Rodney D. Van Meter\\植原 啓介\\三次 仁\\手塚 悟\\高汐 一紀\\武田 圭史\\大越 匡\\} %指導教員
\jhyear  {1}                                   % 平成○年度
\jsyear  {2019}                                 % 西暦○年度
\jkeyword  {表情分析,表情検出,笑顔,表情,人間関係,嗜好分析,データ処理,画像処理,動画処理}     % 論文のキーワード
\jproject{DSFSA(Delta Smile Facial Survey Analyzer)} %プロジェクト名
\mail{massaman@ht.sfc.keio.ac.jp}


% 英語情報(必要なら)
\eclass  {Graduation Thesis}                            % 論文種別
\etitle    {The Establishment of Inclination Bias System Using Feature Points Analysis on Users' Smiles}      % タイトル。改行する場合は\\を入れる
\euniv  {Keio University}                             % 大学名
\efaculty  {Bachelor of Arts in Environment and Information Studies}  % 学部、学科
\eauthor  {Masayoshi Tsuruoka}                           % 著者
\eyear  {2020}                                        % 西暦○年度
\ekeyword  {facial expression analysis, facial expression detection, smile, facial expression, humnan relationships, preference analysis, data processing, image processing, video processing}          % 論文のキーワード
\eproject{DSFSA(Delta Smile Facial Survey Analyzer)}                 %プロジェクト名
\edate{January 2020}


\begin{document}
% 仮綴提出のときは表紙類を外す タイトル, アブストラクト
\ifjapanese
  \jmaketitle    % 表紙(日本語)
\else
  \emaketitle    % 表紙(英語)
\fi

% ■ アブストラクトの出力 ■
%	◆書式:
%		begin{jabstract}〜end{jabstract}	:日本語のアブストラクト
%		begin{eabstract}〜end{eabstract}	:英語のアブストラクト
%		※ 不要ならばコマンドごと消せば出力されない。



% 日本語のアブストラクト
\begin{jabstract}

あぶすとらくと\\
これは最後にまとまってから書くつもり

\end{jabstract}



% 英語のアブストラクト
\begin{eabstract}

abstract\\
English is very very very difficult for me.
I can only use "massamanglish".
That is "99\% body language, and 1 \% passion". 

\end{eabstract}
  % アブストラクト。要独自コマンド、include先参照のこと

\tableofcontents  % 目次
\listoffigures    % 表目次
\listoftables    % 図目次

\pagenumbering{arabic}

\chapter{序論}
\label{chap:introduction}

本章では, はじめに本研究における背景を述べる.ついで, 問題意識を踏まえた上での目的, 製作者の仮説を述べる. 最後に本論文の構成を示す.

\section{背景}
本節では, 本研究の背景としと人と人との繋がりで構成されている現代社会について述べる. ついで人との繋がり形成におけるコミュニケーション上での表情の重要性を述べ, 最後に現代社会における人との関係性を構築する手段が多様化している様子を述べる.

%\begin{quotation}
%\end{quotation} 引用をする際にはこれを使用する

\subsection{人と人との繋がりで構成されている社会}
ヒトは誕生から現代にかけて社会を形成し, その中で集団を作って生活を送ってきた. 社会性をもつ生き物として, 交友関係を広げ, 協力をして日々を生きている. 職場, 学校, 近所, そして家族など人との繋がりは必要不可欠な要素である. 日本人の平均寿命はWHOの調査で84.2歳\cite{WHO_reserch}と言われ,1日1人と出会ったとしても30733人と出会うことになり, 人と関わることなしに生活することは不可能である.
\subsection{コミュニケーションにおける表情の重要性}
人との関係構築をする際にはコミュニケーションが必要不可欠である. コミュニケーションにおいて非言語コミュニケーションがもっとも重要とAlbert Mehrabian\cite{rule_of_Mehrabian}は述べており, 会話中の相手の受け取る情報量は非言語コミュニケーションが93\%を占め, その中でも視覚情報は55\%を占めると言われている. その中でも,人の表情はその人の内面を表しており相手のことを判断するときに重要な判断材料となる.
%表情,笑顔が重要な理由を出す
\begin{figure}[htbp]
    \begin{center}
       \fbox{\includegraphics[width=40mm,bb=0 0 261 261]{mehrabian.jpg}}
    \end{center}
    \caption{メラビアンの法則}
    \label{fig:mehrabian}
\end{figure}


\subsection{人との繋がり形成の多様化}
直接対面でしかなかった出会いの形が, 若い世代を中心に多様化している. オンライン上での出会いが増えてきている. インターネット上で知り合った経験がある10~20代の若者世代は多い.\cite{mandam}目的はオンライン上で友達になったから会う, 友達になりたくて会う, 恋人探しなど様々である.
マッチングアプリなど, 自身のパートナーをインターネット上のデータを基に探すサービス展開も行われており, 登録者は1000万人を超えるものもある結婚相手を探す新しい場を提供する, 婚活パーティーも増えてきており\cite{chane_claire},従来の出会いの方法にくらべ敷居が低くなり, 多種多様な人々との出会いの機会が提供されるようになっている.

\begin{figure}[htbp]
    \begin{center}
       \fbox{\includegraphics[width=100mm,bb=0 0 753 334]{onlinemeeting_to_real.jpg}}
    \end{center}
    \caption{インターネットで知り合った相手にあったことのある若者}
    \label{fig:onlinemeeting_to_real}
\end{figure}


%[20代~50代へのアンケート調査(n=1,091)で ネット上で知り合った人と直接会った割合は 10~20代で約半数
%株式会社マンダム 出会いとデジタルコミュニケーションにおける調査][1]
%[シャンクレールの調査データを使用][2]

\section{問題意識}
前章において, 人との繋がりの重要性および繋がりを形成するコミュニケーションにおける表情の重要性を述べ, 現代社会における出会いの多様性について述べた.
本節では, 本研究おける問題意識について述べる.
\subsection{情報社会における情報過多}
人との繋がり形成の場が多様化し, 出会いの入り口が広くなった分, 情報量の整理がうまくいっていない.
情報過多になっているが故に, 選択肢が多く全ての人にアプローチをして交友関係を構築することは不可能である.
ネット上の出会い調査によると若者が相手に対してギャップを感じた割合は,男女共に70\%であり,
そのギャップにがっかりした割合は男女共に60\%を超えている.\cite{mandam}
つまり, 自分の理想とする人, 自分と考えがあう人と出会える確率は現状まだ低く, また理想や考えがあう人を見逃している可能性が非常に高い.
しかし, 現状はテキストベースで判断をして自分でアプローチをかけたり, アプリケーション側で処理を行いユーザーへ情報提供をすることしかできていない.

%[20代~50代へのアンケート調査(n=1,091)で ネット上で知り合った人と直接会った割合は 10~20代で約半数
%株式会社マンダム 出会いとデジタルコミュニケーションにおける調査][1]

\subsection{相性が悪い人同士における生産性の低下}
コミュニケーションにおいて, 相手と考えが合わない場合精神的な負担が大きくなる.
Mechanicはソーシャルスキルの程度と種々の具体的問題解決能力との2側面から適応行動を規定した上で,適応行動の不足は個体のストレスレベルを上昇させ疾患発症の危険度を高めることを述べた.\cite{Mechanic}
Fisher Beckfield\&McFallは, ソーシャルスキル得点の低い男子大学生が, 高い抑うつ反応を示していたと報告している.\cite{FisherMcFall}
コミュニティーにおいて考えの近い人々をグルーピングをした際にはアイスプレーキングの時間や, 相手の印象を探る時間フェーズを短くすることができ,産出するもののクオリティーの向上や, 時間短縮を見込める.
上記のことより, 考え方の近い人や目的や嗜好が似ている人同士を引き合わせることで活動の効率化, ストレスの軽減に繋がる.


% [David Mechanic, Stress, Illness, and Illness Behavior,Journal of Human Stress, Pages 2-6  09 Jul 2010]
% [Fisher-Beckfield, Denise McFall, Richard M. ,Development of competence inventory for college men and evaluation of relationships between competence and depression, Journal of Consulting and Clinical Psychology, 50(5), 697–705]
%[Associated Newspapers Ltd Part of the Daily Mail, The Mail on Sunday & Metro Media Group の調査][2]
%[Bruno Laeng ,Is Beauty in the Face of the Beholder?,POLOS, July 10,2013 | Volume 8 | Issue 7 ][3]
\subsection{断片的な表情データ}
人の表情は断片的なものではなく連続的なものである. 表情の研究は主に画像処理によって行われており, 中立の表情から笑顔になる過程については分析がほとんど行われていない.
より人間の表出表現を正確に, 繊細に分析する際には連続的に表情の流れの中で分析する必要がある.

\section{目的}
本研究の目的は, 笑顔からお互いを分析し, 人と人との繋がり作成を助長するシステムの作成である.
自身の笑い方と, 魅力を感じる笑い方にはどのような関係性があり, どの顔のパーツ・動きに由来するのかを分析する.
人の考え方・内面は表情に顕著に現れると言われており, Associated Newspapers Ltd Part of the Daily Mailの調査によると目尻と口の動きは相手への信頼性を判断する材料となっている.\cite{TheMailonSunday} Bruno Leangによれば, 似ている人には信頼を置きやすいと述べている.\cite{Bruno}
本研究におけるDelta Smile Facial Survey Analyzer において, ユーザーの笑顔の作り方を数値データとして取得し,
データベース内の笑顔の作り方データを表示し, 順位付けを行うことで笑顔の作り方による嗜好性を分析することが可能である.
本システムの導入により, 人と人との繋がり形成の際により最適な相手をユーザーに表示することができるようなアルゴリズムを作成することが可能になり,
ユーザーにとってよりよい出会いの機会提供を助長することができる.

%[Associated Newspapers Ltd Part of the Daily Mail, The Mail on Sunday & Metro Media Group の調査][2]
%[Bruno Laeng ,Is Beauty in the Face of the Beholder?,POLOS, July 10,2013 | Volume 8 | Issue 7 ]

\section{仮説}
人相学の分野において顔のひとつひとつの形には, その人の性格が表れるとされている.それゆえに, 顔が似ている人は性格が似ていることになる.
また心理学の分野において, Anthony Little[1]は人間同士においては特にカップルに身体的特徴, 表情が似てくる傾向があると述べている.\cite{AnthonyLittle}
さらに, Robert Zajoncは実験で新婚当初に比べ結婚25年の写真ほうがお互いが似ていると結論づけている.\cite{RobertZajonc}
上記の例では好意がお互いの表情を似せていることを述べている.
本研究において, 私は「表情が似ている人には好意を抱きやすい」との仮説を立てる.
William J.Chopikはか飼い主と犬の性格は似ていることを証明した[3]. 犬を迎え入れる際に人は無意識に自分の生活習慣にあった犬に惹かれる傾向があるとしている.
これは人と人との関係性にも言えることなのではないかと考えた.

%Anthony Little, Assortative mating for perceived facial personality traits,Personality and Individual Differences ,Volume 40, Issue 5, April 2006, Pages 973-984
%Robert Zajonc,Convergence in the Physical Appearance of Spouses ,Motivation and Emotion, VoL 11, No. 4, 1987
%William J.Chopik, Old dog, new tricks: Age differences in dog personality traits, associations with human personality traits, and links to important outcomes,
%Journal of Research in Personality,Volume 79, April 2019, Pages 94-108[3]


\section{本文書の構成}

本論文は, 本章を含め全8章で構成する. 本章,第\ref{chap:introduction}章では, 本研究における背景と問題意識, 目的および作成者の仮説を述べた.
第\ref{chap:smile}章では, 関連研究をについてまとめ, 本論文における用語の意味を定義する.
第\ref{chap:aboutDSFSA}章では, 本研究にて作成するシステムDelta Smile Facial Survey Analyzerの概要について説明をする.
第\ref{chap:function}章では, 本システムにおける設計について整理する.
第\ref{chap:developing}章では, 本システムの実装について述べる.
第\ref{chap:pre_experiment}章では, 予備実験について述べる.
第\ref{chap:main_experiment}章では, データ収集および評価実験について述べる.
第\ref{chap:conclusion}章では, 結論および今後の展望について述べる.
%chapであとで管理
%chap 第\ref{chap:introduction}章,第\ref{chap:smile}章
%第\ref{chap:aboutDSFSA}章 第\ref{chap:function}章
%第\ref{chap:developing}章 第\ref{chap:pre_experiment}章
%第\ref{chap:main_experiment}章 第\ref{chap:conclusion}章
  % 序論
%\include{02}  % 本文2
%\include{03}  % 本文3
%\include{04}  % 本文4
\chapter{本研究で使用する笑顔の定義}
\label{chap:smile}

この章では本研究で扱う笑顔について定義する.

\section{表情の分類}

\section{笑顔における顔パーツの動き}

\section{本研究における笑顔の判断}

\section{まとめ}
  % 先行研究まとめ
\chapter{DSFSA(Delta Smile Facial Survey Analyzer)システム}
\label{chap:aboutDSFSA}

この章では本研究で使用するDSFSA(Delta Smile Facial Survey Analyzer)システムについて述べる .

\section{DSFSAシステムの概要}
\section{DSFSAの特徴}
\section{DSFSAの使用方法}
\section{まとめ}
  % システムについて
\chapter{設計}
\label{chap:function}

この章ではDESFAの設計について述べる.

\section{本システムの設計概要}
\section{顔検出モジュール}
\section{笑顔検出モジュール}
\section{画像処理モジュール}
\section{動画作成モジュール}
\section{表示データ作成モジュール}
\section{ランキングデータ取得モジュール}
\section{データ保存モジュール}
\section{笑顔トリミングモジュール}
\section{csvモジュール}
\section{データプロットモジュール}
\section{まとめ}
  % 機能要件
\chapter{実装}
\label{chap:developing}

この章ではDSFSAの実装について述べる.

\section{ユーザーインターフェースの実装}
\subsection{顔検出モジュールの実装}
\subsection{笑顔検出モジュールの実装}
\subsection{画像処理モジュールの実装}
\subsection{動画作成モジュールの実装}
\subsection{表示データ作成モジュールの実装}
\subsection{ランキングデータ取得モジュールの実装}
\subsection{データ保存モジュールの実装}
\subsection{笑顔トリミングモジュールの実装}

\section{分析スクリプトの実装}
\subsection{csvモジュールの実装}
\subsection{データプロットモジュールの実装}
\section{まとめ}
  % 実装
\chapter{予備実験}
\label{chap:pre_experiment}

この章では本研究で行った予備実験について述べる.
本予備実験の目的は, 本実験の際に本システムの処理が妥当であることを証明するためのものとした.
まず笑顔データを作成する際のフレーム数および秒数の指定についてシステムの予備実験を行い,
次に笑顔の動画データを収集する際に使用する採用モデルの評価を行なった.
\section{表情を作るのかかる時間}
本システムはユーザーの顔を認識し, その表情の変化を記録し表情分析を行う.
ユーザーへ笑顔動画データをフィードバックをする際には, 人間が表情の変化を認識することができるフレーム数を
確保しなければならない.
織田らの瞬間的に変化する表情を人がどの程度正確に認知をすることができるのかを検証した研究では,
笑顔と怒りの場合は, 200m\/s まで高い認知をすることが可能であると述べている.\cite{織田朝美2005表情の瞬間的変化の認知}

\subsection{実験}
本システムを起動し, 自身のラップトップPCがどれだけのフレームレート(fps)でユーザーの記録を行うことができるのかを検証する.
取得したfpsの値を用いて, 織田らが明示した1フレーム辺りの表示時間が200ms  以上,
つまりミリフレームレート(fpms)が200以下であることを確認する.

\subsection{結果}
\begin{itemize}
\item 実行結果
\setlength{\parskip}{20pt}
\begin{lstlisting}
start_from_webcam
FPSの設定値、video.get(cv2.CAP_PROP_FPS) : 20.0
\end{lstlisting}
%複数の時は以下のように
%\item 実行コマンド
%\begin{lstlisting}
%$ python dsfsa.py mode_num
%\end{lstlisting}
\end{itemize}
本システムにおけるfpsは20であったため, 1秒間に20フレームを取得することができている.
よって,これをミリフレームレート(fpms)に変換すると,
\begin{equation}
\label{fpms}
 fpms = \frac{1000.0}{20.0} = 50.0
\end{equation}
となる.
よって, 人が認知に必要な200ms以上フレームを取得し, 表示することができているためこのシステムは
人の表情の変化を知覚する条件を満たしていることが証明できた.


\section{OpenCVを使った笑顔検出の妥当性}
本システムでは収集した笑顔動画データを解析する際には, Cambridge大学が開発したオープンソースの
OpenFace を使用している. 顔パーツの位置や, Facial Action Units (以下FAU)によって表情を
判断するパーツの動きの強度を取得することが可能である.
しかし, 開発途中のオープンソースおよび有用なデータを1フレームに対して, 711個データを取得するため
処理時間が長くなってしまう.

pythonの中に含まれる, cProfileを使用して静止画,
1フレームの時間を取得すると, 平均約4秒の処理時間がかかることが判明し, さらに本研究で採用している20フレーム分の処理を行なった場合,
20秒の処理時間を要することが判明した.
以上のことより, 笑顔動画データを作成するためのフレーム取得にはより処理速度の速いOpenCVの中に含まれる
haarcascades\/ haarcascade\_ smile.xmlの笑顔判定モデルを使用した.

 \begin{itemize}
   \setlength{\parskip}{20pt}      %4. 段落間余白
 \item 処理時間時間計測
 \begin{lstlisting}
 $ python -m cProfile -s tottime dsfsa.py
 \end{lstlisting}
 \end{itemize}

\subsection{実験}
OpenCvの笑顔判定モデルを使用して, 人が映っている動画データに対して笑顔検出を行う.
笑顔のフレームを一番最後に5フレーム含み, 全体20秒の笑顔動画データを作成する.
作成した笑顔動画データに対してOpenFaceの処理を行い, 最後のフレームに対して笑顔のFAUである
FAU06(眼窩部眼輪筋)とFAU12(大頬骨筋)の判定値が検出できるかどうかを判定する.
実験はテストデータ3つに対して行った.

\subsection{結果}
OpenCVの笑顔認識モデルを使用して作成した笑顔動画データに対して, OpenFaceの処理を行なった結果を
図\ref{fig:preex1},\ref{fig:preex2},\ref{fig:preex3} に出力した.
赤いグラフがFAU06,12の判定値を示している.
各データともに, FAU06およびFAU12(大頬骨筋)の値を検出することができた.
よってデータ収集の際に, OpenCVの笑顔判別器を使用してデータ収集をすることは可能であることがわかった.

\begin{figure}[htbp]
  \setlength\intextsep{0pt}
    \begin{center}
       \fbox{\includegraphics[width=140mm,bb=0 0 600 427]{preex1.jpg}}
    \end{center}
    \caption{笑顔動画データ1}
    \label{fig:preex1}
\end{figure}

\begin{figure}[htbp]
  \setlength\intextsep{0pt}
    \begin{center}
       \fbox{\includegraphics[width=140mm,bb=0 0 640 427]{preex2.jpg}}
    \end{center}
    \caption{笑顔動画データ2}
    \label{fig:preex2}
\end{figure}
\begin{figure}[htbp]
  \setlength\intextsep{0pt}
    \begin{center}
       \fbox{\includegraphics[width=140mm,bb=0 0 640 427]{preex3.jpg}}
    \end{center}
    \caption{笑顔動画データ3}
    \label{fig:preex3}
\end{figure}

\section{実験の結果・まとめ}
本章では, 本システムが実験をする際の条件を満たしているかの予備実験を行なった.
次章では本システムを用いた評価実験について述べる.
  % 予備実験
\chapter{評価実験}
\label{chap:main_experiment}

この章では本研究で行った評価実験について述べる.

\section{評価実験の概要}
\section{評価実験の目的}
\section{ORF(Open Research Forum)におけるデータ収集}
\section{ユーザーの嗜好分析}

\begin{figure}[htbp]
    \begin{center}
       \fbox{\includegraphics[width=150mm,bb=0 0 1024 768]{choice_myself_fau_for_paper.jpg}}
    \end{center}
    \caption{自身を選択したユーザーのFAUデータ}
    \label{fig:choice_myself_fau_for_paper}
\end{figure}

\begin{figure}[htbp]
    \begin{center}
       \fbox{\includegraphics[width=150mm,bb=0 0 790 1024]{choose_else_for_paper.jpg}}
    \end{center}
    \caption{自身を選択しなかったユーザーのFAUデータ}
    \label{fig:choose_else_for_paper}
\end{figure}

\begin{figure}[htbp]
    \begin{center}
       \fbox{\includegraphics[width=150mm,bb=0 0 1024 115]{usage_guide_fau.jpg}}
    \end{center}
    \caption{ユーザーFAUグラフ凡例}
    \label{fig:usage_guide_fau}
\end{figure}

\section{自身の表情の作り方と好感をもつ笑顔との相関性}
\section{結果}
\section{まとめ}
  % 本実験
\chapter{結論}
\label{chap:conclusion}

この章では本研究における結論について述べる.
まず今後の展望について述べ, 次に具体的なデータの収集・活用方法について述べる,
最後に本研究のまとめを行い,卒業論文とする.

\section{今後の展望}
第\ref{chap:main_experiment}章の評価実験において, 人は一度顔の筋肉を弛緩してから笑顔になるユーザーに対して嗜好傾向がある可能性を示した.
しかし, 本研究では被験者41人, 有効データ36個と非常に傾向分析には少ないデータとなっている.
さらに, データの大半は20代および男性のデータが約68\%と偏りのあるデータとなっている.
今後, 追加データを取得する際には設置場所を考慮し,バランスのとれたデータを取得する必要がある.
また, 本研究は表情ベースでの判断のみになってしまっており, ここの内面状態やどのような基準で
順位づけを行っているのかを考慮することができていない.
Open Research Forum においてユーザー1人,1人へインタビュー調査を行うことは叶わなかったため,
今後は表情ベースの判断に加え, 内面的状態を考慮してデータを提案することが可能なシステム
構成を考える必要がある.

\subsection{TEAM Smileとの連携}
私は慶應義塾大学湘南藤沢キャンパス中澤仁研究室が関与している健康情報コンソーシアムの中にある
Team SMILEに参加し, SmileMeterの開発に携わっている.\cite{teamSMILE}SmileMeterとは自分の笑顔状態を把握し,
自分で安全に保存・蓄積できる、素敵なキオスクデバイスである.
今までに玉川高島屋や, イオンモール幕張新都心, 地域のイベントなどに出展をし,
笑顔と健康の大切さを伝える活動を行っている.
今後,TeamSmileの活動が発展していく上でより多く, 精度の高いデータを集取し活動に活かすために
本システムで作成したOpenFaceを使用した表情解析モジュールをSmileMeterに組み込む予定である.
実証実験の場が今後増えていき, より多くのデータを取得することが見込まれる.
本研究では40人といった小規模実験を行ったが,  より多く幅広い年齢・性別のデータを分析することで
より人と人とを繋ぐシステムを構築するためのデータを取得する.

\begin{figure}[htbp]
    \begin{center}
       \fbox{\includegraphics[width=150mm,bb=0 0 800 450]{takashimaya.jpg}}
    \end{center}
    \caption{玉川高島屋の出展}
    \label{fig:takashimaya}
\end{figure}

\begin{figure}[htbp]
    \begin{center}
       \fbox{\includegraphics[width=150mm,bb=0 0 800 450]{aeon.jpg}}
    \end{center}
    \caption{イオンモール幕張新都心の出展}
    \label{fig:aeon}
\end{figure}

\begin{figure}[htbp]
    \begin{center}
       \fbox{\includegraphics[width=150mm,bb=0 0 800 450]{itabashi.jpg}}
    \end{center}
    \caption{板橋区イベントの出展}
    \label{fig:itabashi}
\end{figure}

\subsection{1月下旬に行った実証実験について}
TeamSMILEの共同研究をしている福岡の株式会社LMOの九州良縁フェスティバルに2020年1月18日鹿児島,
19日熊本に参加した. 本イベントは, 20代以上の独身男女が参加する結婚活動イベントであり,
TeamSMILEはSmileMeterを展示し,来場者にむけて実証実験を行った.
将来のパートナーを決める際には, 自分の運命を大きく作用するため相手選びがより慎重になる.
重要な選択をする際に, 自身や相手の根本的な部分を表情によって分析し, 情報提供をすることができれば
より自分にも相手にとっても良好な人間関係の構築をすることが可能になる.
実際に, SmileMeterを体験したカップルには笑顔が増えシステムがお互いにコミュニケーションの
きっかけとなるような働きをしているように思われた.
今後, 本システムの表情解析モジュールを導入し, 良縁な関係を築いていくきっかけになる,
そしてユーザーごとの引き合わせをすることができるシステムの構築を将来的な展望とする.

\begin{figure}[htbp]
    \begin{center}
       \fbox{\includegraphics[width=150mm,bb=0 0 800 390]{lmo.jpg}}
    \end{center}
    \caption{株式会社LMOとの共同イベント}
    \label{fig:lmo}
\end{figure}

\begin{figure}[htbp]
    \begin{center}
       \fbox{\includegraphics[width=150mm,bb=0 0 800 600]{kyushu.jpg}}
    \end{center}
    \caption{株式会社LMOとの共同イベント当日}
    \label{fig:kyushu}
\end{figure}

\section{本論文まとめ}
本研究は自身の笑顔の作り方と表情ベースによる嗜好判断分析を行った.
41人に対して, 36個の有効笑顔動画データおよびFAU値を取得することができた.
取得したデータの分析により, 人は一度表情の筋肉を弛緩してから笑顔になる笑い方に惹かれている
傾向がみられた.
本システムはより多くのデータを収集, ユーザーの内面状態を組み込むことで
仕事または結婚時のパートナー選択に有益な情報を提供することが可能である.
情報過多になっている現代社会において, ユーザーにとって適切なデータ提供を行い,
より良い人間関係の構築をする機会を提供することが可能になると考えられる.
今後, TeamSmileの活動等を通じて多くのデータを取得し,
本研究, 本システムが人と人とを繋ぐ役割を担うようなシステムになることを期待して卒業論文とする.
  % まとめ


\begin{acknowledgment}
本研究を進めるにあたり,ご指導を頂きました慶應義塾大学環境情報学部教授中澤仁博士に深く感謝いたします.
また,慶應義塾大学環境情報学部教授高汐一紀博士,慶應義塾大学環境情報学部准教授大越匡博士には,
本論文の執筆に当たって御助言を賜りました事を深く感謝致します.
慶應義塾大学中澤研究室の諸先輩方には折りに振れ貴重なご助言を頂きました.
特に慶應義塾大学大学院政策・メディア研究科陳寅特任助教, 慶應義塾大学大学院政策・メディア研究科研究員柘植晃氏,
慶應義塾大学大学院政策・メディア研究科研究員伊藤友隆氏には本論文を執筆するにあたってご指導頂きました.
ここに深く感謝の意を表します.
そして,慶應義塾大学大学院博士佐々木航氏,慶應義塾大学大学院博士課程磯川直大氏,
慶應義塾大学大学院博士課程井村和博氏,慶應義塾大学大学院博士課程河崎隆文氏には,本
研究に対し,多くの時間を割いていただきご指導を頂きました.
ここに深く感謝の意を表します.
グループの垣根を超えて゚様々な助言をいただいた栄元優作氏,本木悠介氏,三上量弘氏,小澤遼氏
特にメンターとして様々な助言,ご指導いただきました片山晋氏に心よりに深く感謝致します.
また野田悠加氏, 菅原メリッサ沙良氏, 山根卓氏をはじめとする研究室の後輩たち, 特にグループHAISYSのメンバーには様々な面でご協力をいただきました.
そして,同じ研究室の同期として様々な助言をしていただいた,海法修平氏,柿野優衣氏,川島寛乃氏,谷村朋樹氏,羽柴彩月氏,中村拓朗氏,沼本奨太郎氏,山田佑亮氏,
勝又健登氏,姜欣怡氏に深く感謝致します.
最後に,大学4年間に渡る生活を支えてくれた家族に感謝致します.\\
\begin{flushright}
2020年1月28日\\
鶴岡 雅能
\end{flushright}

\end{acknowledgment}
  % 謝辞。要独自コマンド、include先参照のこと
\include{91_bibliography}  % 参考文献。要独自コマンド、include先参照のこと
\appendix
\chapter{付録の例}

付録?何描こうか
それこそ野中研究室のこと書くのもあり?

\section{見出し1}

見出しの中だよ

\subsection{セクションわけ1-1}
セクションわけされた中だよ

\section{見出し2}

見出しの中だよ

\subsection{セクションわけ2-1}

セクションわけされた中だよ
    % 付録

\end{document}
