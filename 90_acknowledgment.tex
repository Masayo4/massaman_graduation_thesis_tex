\begin{acknowledgment}
本研究を進めるにあたり,ご指導を頂きました慶應義塾大学環境情報学部教授中澤仁博士に深く感謝いたします.
また,慶應義塾大学環境情報学部教授高汐一紀博士,慶應義塾大学環境情報学部准教授大越匡博士には,
本論文の執筆に当たって御助言を賜りました事を深く感謝致します.
慶應義塾大学中澤研究室の諸先輩方には折りに振れ貴重なご助言を頂きました.
特に慶應義塾大学大学院政策・メディア研究科陳寅特任助教, 慶應義塾大学大学院政策・メディア研究科研究員柘植晃氏,
慶應義塾大学大学院政策・メディア研究科研究員伊藤友隆氏には本論文を執筆するにあたってご指導頂きました.
ここに深く感謝の意を表します.
そして,慶應義塾大学大学院博士佐々木航氏,慶應義塾大学大学院博士課程磯川直大氏,
慶應義塾大学大学院博士課程井村和博氏,慶應義塾大学大学院博士課程河崎隆文氏には,本
研究に対し,多くの時間を割いていただきご指導を頂きました.
また,研究室での活動をする上で, 書類手続き,交通手段の手配等事務関連をサポートしてくださった秘書の松尾さん, 遠藤さんへ
ここに深く感謝の意を表します.
グループの垣根を超えて゚様々な助言をいただいた栄元優作氏,本木悠介氏,三上量弘氏,小澤遼氏,
特にメンターとして様々な助言,ご指導いただきました片山晋氏に心より深く感謝致します.
また野田悠加氏, 菅原メリッサ沙良氏, 山根卓氏をはじめとする研究室の後輩たち, 特にグループHAISYSのメンバーには様々な面でご協力をいただきました.
そして,同じ研究室の同期として様々な助言をしていただいた,海法修平氏,柿野優衣氏,川島寛乃氏,谷村朋樹氏,羽柴彩月氏,中村拓朗氏,沼本奨太郎氏,山田佑亮氏,
勝又健登氏,姜欣怡氏に深く感謝致します.
最後に,大学4年間に渡る生活を支えてくれた家族に感謝致します.\\
\begin{flushright}
2020年1月28日\\
鶴岡 雅能
\end{flushright}

\end{acknowledgment}
